\documentclass[a4paper]{jpconf}
%\documentclass[aps,pra,onecolumn,superscriptaddress,english,showpacs]{revtex4-1}

\usepackage{natbib}
\usepackage[utf8]{inputenc}
\usepackage[T1]{fontenc}
\usepackage[english, french]{babel} %français
\usepackage{amsmath}
\usepackage{amsfonts}
\usepackage{makeidx}
\usepackage{graphicx}
%\usepackage[left=2cm,right=2cm,top=2cm,bottom=2cm]{geometry}
\usepackage{mathtools} %dcases
\usepackage{braket} %quantum mechanics
\usepackage[linkcolor=black, citecolor=black]{hyperref} % hyperlinks
\usepackage[usenames,dvipsnames]{xcolor} % colors for text

\usepackage{tikz} % drawing in LaTeX
\usetikzlibrary{positioning}
\usepackage{ dsfont } % hollow letters

% compile child files with separate preambles, and include them in the document
\usepackage{standalone}

% subfigure
%\usepackage{subcaption}

% the equal sign I use to define something
\newcommand{\define}{\ensuremath{ \overset{\text{def}}{=} }}

\begin{document}

\title{Gap structure and topological indices on the Fibonacci quasicrystal.}
\author{Nicolas Macé, Anuradha Jagannathan and Frédéric Piéchon}
\address{Laboratoire de physique des Solides, Université Paris-Saclay, 91400 Orsay, France}
\ead{nicolas.mace@u-psud.fr}

\selectlanguage{english}

\date{\today}

\begin{abstract}
\end{abstract}


\emph{Introduction}

\section{The Fibonacci quasicrystal}

\subsection{The geometrical model}

\subsection{Electrons on the Fibonacci chain}

Cite experiments: Eric.

\section{The gap labelling theorems}

\subsection{Gap labelling of the infinite quasicrystal}

\subsection{Gap labelling of the approximants}

\subsection{Gap labelling in the strong modulation limit}

\section{Gap labels as topological indices}

\section{Permanent and transient gaps}

\section{Gap indices and gap widths}

\bibliographystyle{unsrt}
\bibliography{bib}
\end{document}