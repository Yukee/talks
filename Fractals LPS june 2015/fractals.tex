\documentclass[xcolor=x11names,compress,professionalfonts]{beamer}

%% General packages %%%%%%%%%%%%%%%%%%%%%%%%%%%%%%%%%%
\usepackage[utf8]{inputenc}
\usepackage{graphicx}
\usepackage{tikz}
\tikzset{% change default arrow tips
    >=latex
}
\usepackage{ifthen}

\usepackage{amsmath}
\usepackage{nicefrac}

%%%%%%%%%%%%%%%%%%%%%%%%%%%%%%%%%%%%%%%%%%%%%%%%%%%%%%


%% Beamer Layout %%%%%%%%%%%%%%%%%%%%%%%%%%%%%%%%%%
\useoutertheme[subsection=false,shadow]{miniframes}
\useinnertheme{rectangles}

\setbeamertemplate{navigation symbols}{}%remove navigation symbols

\usepackage{libertine}
\usepackage[T1]{fontenc}

\setbeamerfont{title like}{shape=\scshape}
\setbeamerfont{frametitle}{shape=\scshape}

\setbeamercolor*{lower separation line head}{bg=DeepSkyBlue4} 
\setbeamercolor*{normal text}{fg=black,bg=white} 
\setbeamercolor*{alerted text}{fg=red} 
\setbeamercolor*{example text}{fg=black} 
\setbeamercolor*{structure}{fg=black} 
 
\setbeamercolor*{palette tertiary}{fg=black,bg=black!10} 
\setbeamercolor*{palette quaternary}{fg=black,bg=black!10} 

\renewcommand{\(}{\begin{columns}}
\renewcommand{\)}{\end{columns}}
\newcommand{\<}[1]{\begin{column}{#1}}
\renewcommand{\>}{\end{column}}
%%%%%%%%%%%%%%%%%%%%%%%%%%%%%%%%%%%%%%%%%%%%%%%%%%

\usepackage{braket}

%%%My Math

\newcommand{\pd}[2]{\frac{\displaystyle \partial #1}{\displaystyle\partial #2}} % for partial derivatives
\newcommand{\dx}{\mathrm{d}x}
\renewcommand{\d}[1]{\mathrm{d}#1}
\newcommand{\nth}{$n^\text{th}$ }

\newcommand{\mean}[1]{\langle #1 \rangle}
\DeclareMathOperator{\Pf}{Pf}
\DeclareMathOperator{\Tr}{Tr}

\begin{document}


\begin{frame}
\title{Fractals and physics}
%\subtitle{SUBTITLE}
\author{ Nicolas Macé (inspired by E. Akkermans) }
\date{
	June 2, 2015
}
\titlepage
\end{frame}


\begin{frame}{Outline}
    \tableofcontents[hideallsubsections]
\end{frame}

\section{Fractals and their geometry}
%Each section needs a subsection for the small points on top to show up
\subsection{Dummy}

\begin{frame}{Symmetry}
    \begin{itemize}
        \item Euclidean space (continuous translational symmetry \& continuous scaling symmetry)
            \begin{itemize}
                \item Crystallographic lattice (discrete translational symmetry)
                \item Fractal set/manifold (discrete scaling symmetry)
            \end{itemize}
    \end{itemize}
    
    $\rightarrow$ fractal: infinitely divisible object
    \includegraphics[scale=1.00]{sierpinski.pdf}
\end{frame}

\begin{frame}{Geometric construction}{A first example: the Sierpiński triangle}

\begin{itemize} 

\begin{columns}
\newcommand{\s}{.2}
  \begin{column}{3cm}
    \includegraphics[scale=\s]{sierpinski0.pdf}
  \end{column}

  \begin{column}{3cm}
     \includegraphics[scale=\s]{sierpinski1.pdf}
  \end{column}
  
  \begin{column}{3cm}
    \includegraphics[scale=\s]{sierpinski2.pdf}
  \end{column}
  ...
    \begin{column}{3cm}
    \includegraphics[scale=\s]{sierpinskiInfty.pdf}
  \end{column}
\end{columns}

 	\item independant of starting shape $\rightarrow$ only determined by the geometrical transformations used.

\begin{columns}
\newcommand{\s}{.2}
  \begin{column}{3cm}
    \includegraphics[scale=\s]{sierpinksi_ball_0.pdf}
  \end{column}

  \begin{column}{3cm}
     \includegraphics[scale=\s]{sierpinksi_ball_1.pdf}
  \end{column}
  
  \begin{column}{3cm}
    \includegraphics[scale=\s]{sierpinksi_ball_2.pdf}
  \end{column}
  ...
    \begin{column}{3cm}
    \includegraphics[scale=\s]{sierpinksi_ball_Infty.pdf}
  \end{column}
\end{columns}

	\item The Sierpiński triangle is constructed by an Iterated Function System (IFS).
\end{itemize}

\end{frame}

\begin{frame}{Geometric construction}{Iterated function systems}

\begin{columns}
\newcommand{\s}{.2}

  \begin{column}{1.5cm}
  \centering
    \includegraphics[scale=0.04]{leaf1.png}
  \end{column}

  \begin{column}{3cm}
  \centering
    \includegraphics[scale=0.08]{leaf.png}
  \end{column}

  \begin{column}{3cm}
  \centering
     \includegraphics[scale=.4]{fern.png}
  \end{column}
\end{columns}

\begin{itemize}
	\item Every fractal is approached by an IFS (Barnsley).
\end{itemize}

\end{frame}

\begin{frame}{Geometric construction}{``non-trivial'' fractals: Julia sets}
\begin{itemize}
	\item Define a recurrence $z_{n+1} = z_n^2 + c$
	\item Julia set: boundary of the convergence domain
\end{itemize}

\centering
\only<1>{
	\includegraphics[scale=0.2]{julia.png}
	
	\scriptsize
	    Julia set $c = -0.77 + 0.22 i$
}
    
\only<2>{
	\includegraphics[scale=0.16]{julia2.png}
	
	\scriptsize
	    Julia set $c = -0.39 - 0.59 i$
}
\end{frame}

\section{Physics on fractals: fractal dimensions}
\subsection{Dummy}
\begin{frame}{Scaling}

\begin{itemize}
	\item Give a physical meaning $\rightarrow$ give a length scale

\begin{columns}
\newcommand{\s}{.2}
  \begin{column}{3cm}
  \centering
    \includegraphics[scale=\s]{scale0.pdf}
  \end{column}

  \begin{column}{3cm}
  	\centering
     \includegraphics[scale=\s]{scale1.pdf}
  \end{column}
  
  \begin{column}{1cm}
  \centering
  ...
  \end{column}
  
    \begin{column}{3cm}
    \centering
    \includegraphics[scale=\s]{scaleInfty.pdf}
  \end{column}
\end{columns}

	\item A natural way of doing it:
	
	{\centering 
	\includegraphics[scale=0.4]{sierpinsky_nature.jpg}
	\scriptsize
	
	Assembling molecular Sierpiński triangle fractals, Nature Chemistry (2015)
	
	}
	
	\item Scaling of physical quantities? 
	\begin{itemize}
		\item $M(L) \propto L^d$ on a $d$-dimensional Euclidean manifold... What happens on a fractal one?
	\end{itemize}
\end{itemize}
	
\end{frame}

\begin{frame}{The mass dimension}

\begin{itemize}
	\item $M(L) \propto L^d$ on a $d$-dimensional Euclidean manifold... What happens on a fractal one?

\begin{columns}
\newcommand{\s}{.2}
  \begin{column}{3cm}
  	\centering
    \includegraphics[scale=\s]{scale0.pdf}
    \scriptsize
    \[ M_0 \]
  \end{column}

  \begin{column}{3cm}
  	\centering
     \includegraphics[scale=\s]{scale1.pdf}
     \scriptsize
    \[ M_1= 3M_0 \]
  \end{column}
  
  \begin{column}{1cm}
  \centering
  ...
  \end{column}
  
    \begin{column}{3cm}
    \centering
    \includegraphics[scale=\s]{scaleInfty.pdf}
    \scriptsize
    \[ M_n= 3^nM_0 \]
  \end{column}
\end{columns}

\[ M(L) \propto L^{d_M} \text{, with } d_M = \log 3/\log 2 \]

	\item $d_M$ is the mass (or Hausdorff) dimension.
	
	\item $1 < d_M < 2$ different from $d = 1$, non-integer $\rightarrow$ signature of a fractal manifold.

\end{itemize}

\end{frame}

\begin{frame}

\begin{itemize}
\item Mass dimension: spot and characterize fractals, from large scales... 

	{\centering
    \includegraphics[scale=.18]{combes_interstellar_medium.png}
    \scriptsize
    
    [Taken from \textit{Astrophysical Fractals: Interstellar Medium and Galaxies}]
    
    }
\item  ... to small ones

  	\centering
    \includegraphics[scale=.3]{crumpled_paper.png}
    \scriptsize
\end{itemize}
\end{frame}

\begin{frame}{The electric dimension}

\begin{columns}
\newcommand{\s}{.2}
  \begin{column}{3cm}
  	\centering
    \includegraphics[scale=\s]{res0.pdf}
    \scriptsize
    \[ R_0 \]
  \end{column}

  \begin{column}{3cm}
  	\centering
     \includegraphics[scale=\s]{res1.pdf}
     \scriptsize
    \[ R_1= ? R_0 \]
  \end{column}
  
  \begin{column}{1cm}
  \centering
  ...
  \end{column}
  
    \begin{column}{3cm}
    \centering
    \includegraphics[scale=\s]{resInfty.pdf}
    \scriptsize
    \[ R_n= ?^nR_0 \]
  \end{column}
\end{columns}

\[ R(L) \propto L^{d_e} \text{, with } d_e = \log ?/\log 2 \]

\begin{itemize}
	\item $d_M \neq d_e$, and they both reflect the structure of the fractal manifold.
	\item On an Euclidean manifold $d_M$ and $d_e$ would have been independant its structure, they would have only depended on $d$, its dimension.
\end{itemize}

\end{frame}

\begin{frame}{Black body spectrum: the spectral dimension}

\begin{itemize}
	\item Euclidean cavity (manifold)
	\begin{itemize}
		\item number of spectral modes independant of the cavity shape
		\[ N(k) \propto V k^d\]
	\end{itemize}
	
	\item Fractal cavity
	\begin{itemize}
		\item number of spectral modes independant of the cavity shape
		\[ N(k) \propto V_s k^{d_M} \]
	\end{itemize}
\end{itemize}
    
\end{frame}

\begin{frame}{The Fibonacci chain: hidden fractals}
    \begin{itemize}
    	\item Fibonacci chain: no obvious fractal nature, but...
    \end{itemize}
\end{frame}

\begin{frame}{Interlude: discrete and continuous scale invariance}
    \begin{columns}
\newcommand{\s}{.2}
  \begin{column}{5cm}
  	\begin{block}{Continuous scale invariance}
  	\begin{align*}
		\forall a,~ f(ax) &= b(a) f(x) \\
		\text{Then } f(x) &= C x^\alpha
	\end{align*}
  	\end{block}
  \end{column}

  \begin{column}{5cm}
  	\begin{block}{Discrete scale invariance}
  	\begin{align*}
		\exists a,~ f(ax) &= b(a) f(x) \\
		\text{Then } f(x) &= ?
	\end{align*}
  	\end{block}
  \end{column}
\end{columns}
A tool to analyze scale invariant functions: the Mellin tranform.
\[ \{ f \}(z) = \int_{\mathbb{R}^+} f(x) x^{-z} \frac{\d{x}}{x} \]
\[ \left\{ x^\alpha \right\}(z) = \frac{1}{z + \alpha} \]
$\rightarrow$ the scaling factor(s) of a function are given by the poles of its Mellin transform.
\end{frame}

\section{Summary}
\subsection{Dummy}

\begin{frame}{Summary}

    \begin{itemize}
        \item 4 $\mathbb{Z}_2$ invariants $\Rightarrow$ 16 classes
        \item weak $\leftrightarrow$ strong
        \item general description of any crystaline surface by Miller indices
    \end{itemize}
\end{frame}

\begin{frame}{Outlook}
    Experimental realizations

    Materials
    \begin{itemize}
        \item Bi$_{x}$Sb$_{1 - x}$
        \item Bi$_2$Se$_3$
        \item Bi$_2$Te$_3$
        \item HgTe - under strain
    \end{itemize}
    
    
\end{frame}

\begin{frame}{References}

    \begin{itemize}
        \item Fu, Kane, Mele ``Topological Insulators in Three Dimensions'' (2007) PhysRevLett.98.106803
        \item Moore, Balents ``Topological invariants of time-reversal-invariant band structures'' (2007) PhysRevB.75.121306
        \item Bernevig, Hughes ``Topological Insulators and Topological Superconductors'' (2013) Princeton University Press
        \item Franz, Molenkamp ``Topological Insulators'' (2013) Springer
    \end{itemize}

\end{frame}

\end{document}
